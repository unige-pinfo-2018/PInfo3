\documentclass[12pt,a4paper,oneside, titlepage]{article}

\usepackage[utf8]{inputenc}
\usepackage[T1]{fontenc}
\usepackage[french]{babel}
\usepackage{amssymb}
\usepackage{amsmath}
\usepackage{graphicx}
\usepackage{float}
\usepackage{mathtools}
\usepackage{enumitem}
\usepackage{url}
\setitemize{noitemsep,topsep=10pt,parsep=10pt,partopsep=0pt}
\usepackage{authblk}

\setlength\parindent{0pt}

\renewcommand{\familydefault}{\sfdefault}

\graphicspath{ {images/} }


% BEGIN FRONT PAGE ****************************************************
\title {Semester Project Vision Document  \\ \large Team MOKATH}

\author{Matteo BESANCON}
\author{Heloy ESTEVANIN LEAL}
\author{Amir HOSSEIN HEIDARI}
\author{Ornela TCHAWOU BILLY}
\author{Terry VOGELSANG}
\author{Karine ZUERCHER}

\affil{Centre Universitaire D'Informatique, University Of Geneva}

\renewcommand\Authands{ and }

\date{\today}

% END FRONT PAGE ****************************************************


\begin{document}

	\renewcommand{\labelitemi}{$\bullet$}
	\maketitle
	\tableofcontents
	\newpage
	
% INTRO SECTION $$$$$$$$$$$$$$$$$$$$$$$$$$$$$$$$$$$$$$$$$$$$$$$$$$$$$$$$$$$$$$$$$$$$$$$$$$$$$$$$$$$$$$$$$$

	\section{Introduction}
	
		\subsection{Document's goals}
		\subsection{Coverage}
		    This document covers the requirements and specifications of the semester project
		    taking place during Spring Semester 2018 @ UNIGE.
		    
		\subsection{Definition, Acronyms and Abbreviations}
		
		\begin{itemize}
		
		    \item \textbf{Student} A University student
		    \item \textbf{Domain expert} Could be a student at ease in the domain, a teacher or a teaching 
		    assistant
		    
		\end{itemize}
	
		\subsection{Global view of the document}
		
% POSITION SECTION $$$$$$$$$$$$$$$$$$$$$$$$$$$$$$$$$$$$$$$$$$$$$$$$$$$$$$$$$$$$$$$$$$$$$$$$$$$$$$$$$$$$$$$$$$

	\section{Position}		
		
		% COMMERCIAL OPPORTUNITY ****************************************************
		\subsection{Commercial opportunity}
		    The growing usage of internet platforms in the domain of knowledge sharing
		    represent a great business opportunity targeting all university student accross 
		    the world.
		    
		    
		% PROBLEM POSITIONING ***************************************************
		\subsection{Problem Positioning}
		
         \subsubsection{Problem}
         When students encounter an issue and doesn't find any help online, the only solutions at their disposal
         is to contact a teacher or another student who is at ease in the domain, these are referred as domain 
         expert in this document. The problem is the alternatives rely on teachers and assitant teachers 
         availability and social affinities with other students.
         
         \subsubsection{Affected entities}
         The problem affects students who encounter issues in a certain domain during their University 
         revisions / exercises assessments.
            
         \subsubsection{Problem Impact}
         The problem could impact students grades and results during their academic cursus. 
                  
         \subsubsection{Possible satisfying solution}
         Provide an online platform to ease the contact between students needing help and those who could help 
         them because of their good understanding of the problematic domain. The idea is alos to provide a 
         centralized place where teachers could publish solutions, tips and tricks making them widely available 
         through the students community.
         
         
         % PRODUCT POSITIONING ***************************************************
		\subsection{Product Positioning}
		
		\subsubsection{Audience}
		As said before, the targeted audience includes students who encounter issues in a certain domain during 
		their University revisions / exercises assessments but also teachers who would be able to publicly and 
		widely share their advices, tips and tricks.
		
	    \subsubsection{Opportunity and Needs}
		The combination between high demande in online knowledge sharing platforms and the problem encountered 
		by students during their studies tends to open a lot of business opportunities.
		
	    \subsubsection{The Platform}
		The online platform would fulfill the needs in communication between students and domain experts. It 
		would ease the sharing of high-valued knowledge and offer a centralized place to browse when 
		encountering an issue. The platform would be distributed as a service to all Universities.
		
	    \subsubsection{Competition}
		The main competitors of the project are companies and platforms like Quora~\cite{quoraWebsite}, 
		StackExchange~\cite{stackExchangeWebsite} and MOOCS platforms such as Udemy~\cite{udemyWebsite}, 
		Coursera~\cite{courseraWebsite} and more ... \newline
		
		The difference between our project and the competition stays in the flexible but adapted scope of our 
		project. \newline 
		
		Each student will be able to choose between global or local pools of domain experts allowing him to 
		choose wether he wants to communicate with people from his establishment or with people in all world's 
		universities. \newline
		
		A rewarding system will ensure engagement of other local students. After a certain amount 
		of local answers, a student can be rewarded with credits (For example ECTS credits in the European 
		Credit Transfer and Accumulation System~\cite{ectsWiki}) that he 
		can use to validate work in his degree cursus. 
		
% USERS AND STAKEHOLDERS SECTION $$$$$$$$$$$$$$$$$$$$$$$$$$$$$$$$$$$$$$$$$$$$$$$$$$$$$$$$$$$$$$$$$$$$$$$$
	\section{Users and Stakeholders Description}
	\subsection{Market Size}
	\subsection{Stakeholders}
	\subsection{Users}
	\subsection{User Environment}
	\subsection{Stakeholders Profiles}
	\subsection{Users Profiles}
	\subsection{Users \& Stakeholders Key Requirements}
	\subsection{Competition and Alternatives}

    \newpage
    \bibliography{refs}{}
    \bibliographystyle{plain}
\end{document}
