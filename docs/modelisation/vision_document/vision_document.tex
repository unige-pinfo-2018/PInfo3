\documentclass[12pt,a4paper,oneside, titlepage]{article}

\usepackage[utf8]{inputenc}
\usepackage[T1]{fontenc}
\usepackage[english]{babel}
\usepackage{amssymb}
\usepackage{amsmath}
\usepackage{graphicx}
\usepackage{float}
\usepackage{mathtools}
\usepackage{enumitem}
\usepackage{url}
\setitemize{noitemsep,topsep=10pt,parsep=10pt,partopsep=0pt}
\usepackage{authblk}
\usepackage{titlesec}
\titleformat{\section}{\normalfont\Large\bfseries}{}{0pt}{}

\setlength\parindent{0pt}

\renewcommand{\familydefault}{\sfdefault}

\graphicspath{ {images/} }


% BEGIN FRONT PAGE ****************************************************
\title {Semester Project Vision Document  \\ \large Team MOKATH}

\author{Matteo BESANCON}
\author{Heloy ESTEVANIN LEAL}
\author{Amir HOSSEIN HEIDARI}
\author{Ornela TCHAWOU BILLY}
\author{Terry VOGELSANG}
\author{Karine ZUERCHER}

\affil{Centre Universitaire D'Informatique, University Of Geneva}

\renewcommand\Authands{ and }

\date{\today}

% END FRONT PAGE ****************************************************


\begin{document}

	\renewcommand{\labelitemi}{$\bullet$}
	\maketitle
	\tableofcontents
	\newpage
	
% REVISION HISTORY SECTION $$$$$$$$$$$$$$$$$$$$$$$$$$$$$$$$$$$$$$$$$$$$$$$$$$$$$$$$$$$$$$$$$$$$$$$$$$$$$$$$$$$$$$$$$$

		\section{Revision History}
		
		\begin{center}
		\def\arraystretch{1.5}
            \begin{tabular}{| c | c | c | c |}
            \hline
            Date & Version & Description & Author \\ \hline
            March 13-14th 2018 & 0.1 & Initial Work  & Terry VOGELSANG \\ \hline
            \end{tabular}
        \end{center}
        
        \newpage
        
% INTRO SECTION $$$$$$$$$$$$$$$$$$$$$$$$$$$$$$$$$$$$$$$$$$$$$$$$$$$$$$$$$$$$$$$$$$$$$$$$$$$$$$$$$$$$$$$$$$

	\section{Introduction}
	
		\subsection{Document's goals}
		\subsection{Coverage}
		    This document covers the requirements and specifications of the semester project
		    taking place during Spring Semester 2018 @ UNIGE.
		    
		\subsection{Definition, Acronyms and Abbreviations}
		
		\begin{itemize}
		
		    \item \textbf{Domain Expert} Could be a student at ease in the domain, a teacher or a teaching assistant 
		    \item \textbf{Student} A Higher Education Institution student
		    \item \textbf{Teacher} A Higher Education Institution teacher
		    \item \textbf{Institution} A Higher Education Institution
		    \item \textbf{Assistant Teacher} A Higher Education Institution assistant teacher
		    \item \textbf{Local} Person or people from the same institution
		  
		\end{itemize}
	
		\subsection{Global view of the document}
		
% POSITION SECTION $$$$$$$$$$$$$$$$$$$$$$$$$$$$$$$$$$$$$$$$$$$$$$$$$$$$$$$$$$$$$$$$$$$$$$$$$$$$$$$$$$$$$$$$$$
    \newpage
	\section{Position}		
		
		% COMMERCIAL OPPORTUNITY ****************************************************
		\subsection{Commercial opportunity}
		    The growing usage of internet platforms in the domain of knowledge sharing
		    represent a great business opportunity targeting all university student accross 
		    the world.
		    
		    
		% PROBLEM POSITIONING ***************************************************
		\subsection{Problem Positioning}
		
         \subsubsection{Problem}
         When students encounter an issue and doesn't find any help online, the only solutions at their disposal
         is to contact a teacher or another student who is at ease in the domain, these are referred as domain 
         expert in this document. The problem is the alternatives rely on teachers and assitant teachers 
         availability and social affinities with other students.
         
         \subsubsection{Affected entities}
         The problem affects students who encounter issues in a certain domain during their University 
         revisions / exercises assessments.
            
         \subsubsection{Problem Impact}
         The problem could impact students grades and results during a student academic cursus. 
                  
         \subsubsection{Possible satisfying solution}
         Provide an online platform to ease the contact between students needing help and those who could help 
         them because of their good understanding of the problematic domain. The idea is alos to provide a 
         centralized place where teachers could publish solutions, tips and tricks making them widely available 
         through the students community.
         
         
         % PRODUCT POSITIONING ***************************************************
		\subsection{Product Positioning}
		
		\subsubsection{Audience}
		As said before, the targeted audience includes students who encounter issues in a certain domain during 
		their University revisions / exercises assessments but also teachers who would be able to publicly and 
		widely share their advices, tips and tricks.
		
	    \subsubsection{Opportunity and Needs}
		The combination between high demande in online knowledge sharing platforms and the problem encountered 
		by students during their studies tends to open a lot of business opportunities.
		
	    \subsubsection{The Platform}
		The online platform would fulfill the needs in communication between students and domain experts. It 
		would ease the sharing of high-valued knowledge and offer a centralized place to browse when 
		encountering an issue. The platform would be distributed as a service to all Universities.
		
	    \subsubsection{Competition}
		The main competitors of the project are companies and platforms like Quora~\cite{quoraWebsite}, 
		StackExchange~\cite{stackExchangeWebsite} and MOOCS platforms such as Udemy~\cite{udemyWebsite}, 
		Coursera~\cite{courseraWebsite} and more ... \newline
		
		The difference between our project and the competition stays in the flexible but adapted scope of our 
		project. \newline 
		
		Each student will be able to choose between global or local pools of domain experts allowing him to 
		choose wether he wants to communicate with people from his establishment or with people in all world's 
		universities. \newline
		
		A rewarding system will ensure engagement of other local students. After a certain amount 
		of local answers, a student can be rewarded with credits (For example ECTS credits in the European 
		Credit Transfer and Accumulation System~\cite{ectsWiki}) that he 
		can use to validate work in his degree cursus. 
		
% USERS AND STAKEHOLDERS SECTION $$$$$$$$$$$$$$$$$$$$$$$$$$$$$$$$$$$$$$$$$$$$$$$$$$$$$$$$$$$$$$$$$$$$$$$$
    \newpage
	\section{Users \& Stakeholders Description}
	\subsection{Market Size}
	The key elements in the market that motivates the product development are based on a total absence of any platform 
	targeting knowledge sharing integration in the academic cursus. As said before, the increasing usage of online
	platforms as a knowledge source extends even more the targeted audience. \newline
	
	According to Eurostat, in the EU-28, there were 19.5 million tertiary education students in 
	2015~\cite{tertiaryStudentsEU} repartited in 2465 higher education institutions. \newline
	
	These datas would be an approximate market size in Europa only. Taking the whole rest of the world, it brings us to about 207 million of students 
	around the world according to UNESCO~\cite{higherEducationPaper}
	
	\subsection{Stakeholders}
	    \subsubsection{Higher Eductation Institutions}
	        The higher education institutions represent the main partners of our product. They won't be directly in
	        contact with the system but will take the role of the client. Their interest is to subscribe to a reliable
	        and ergonomic system that provides knowledge sharing capabilities accross the institution. The 
	        adminsitration of these institutions will be our main point of contact with our users. They'll serve as an
	        intermediary between the final users and our company who offers the product. They will also assume the
	        exchange between proprietary points and cursus credits according to predefined policies on the website. 
	    
	    \subsubsection{Our Company : MOKATH}
	        Our company is in charge of the support, sales and development of the product. The MOKATH company is 
	        responsible to ensure the maintainability, the security and the compliance of the system and product.
	        The company is also in charge of controlling and managing the development process in order to deliver the
	        products in time. The company must also make sure that there exists a market opportunity.
	        
	\subsection{Users}
	    \subsubsection{Student}
	    The student as a user, will need to be able to both ask and answer questions on the platform. He'll also need to 
	    be able to manage his account settings and informations but also to manager the exchange between his earned
	    platform credits and cursus credits.
	    
	    	\subsubsection{Teacher}
	    	The teacher as a user, will need to be able to answer questions, create course tracks, and certify domain experts 
	    	in his domain. He'll also need to be able to manage his account settings and informations.

	    \subsubsection{Assistant Teacher}
	     The assistant teacher as a user, will need to be able to answer questions, create course tracks. He'll also
	     need to be able to manage his account settings and informations.
	     
        \subsubsection{Institution Administrator}
        The institution administrator as a user, will need to be able to validate credits exchange. He'll also
	    need to be able to manage the institution account settings and informations. This person is the intermediary
	    between the institution and the platform.
	    
	    
	\subsection{User Environment}
	The user environment will comprise both web browser and mobile. Instead of developing a dedicated mobile app, a 
	responsive web app will be implemented in order to make the app accessible on both mobile and desktop hosts. \newline
	
	Users will pass most of the time writing questions and answers. The user might be able to login via his Facebook or 
	Google Account.
	
	\newpage 
	
	\subsection{Stakeholders Profiles}
	    \subsubsection{Higher Education Institutions}
	    These stakeholders represent the clients. More specifically, the direct point of contact will be the Institution
	    Administrator who's going to use our platform to make the Insitution available on the platform. The institution 
	    adminsitrator will probably be either the community manager of the institutions or an administrative. They are 
	    expected to know how to use a web interface. The institution's interests are the fact that our platform can bring
	    comfort and socialization to its campus. The deliverable expected by these stakeholders is the final web app. The
	    success of the project largely depends on these stakeholders engagements.
	    
	    \subsubsection{Our Company : MOKATH}
	    Our company represent the developer, vendor and maintainer of the platform. The company is responsible to make the 
	    platform available to its client. The success criteria is directly measured with the spreading, the usage and the 
	    consumer satisfaction of the platform accross higher education institutions. The deliverable must be the final web
	    app and a global after-sale support including client support, FAQ and maintenances.
	    
	\subsection{Users Profiles}
		\subsubsection{Students}
	    The students can both represent people needing help or people answering questions. The students activities on the
	    platform will mainly include writing questions and answers. A student is probably between 18-30 years old and can 
	    be from any country in the world. These stakeholders should have sufficient knowledge in the domaine of web
	    applicationusage as during an higher education cursus, you're often required to interact with web applications 
	    services. The student must provide a valid student ID if he wants to register to its local institution domain if 
	    he wants to benefit from local knowledge and credits rewards. Though, even officially verified, he'll still be
	    able to post questions and answers anonymously from others users points of view. The expected delivrable by the 
	    student is a web app that complies to his needs (Post questions, answer questions, get rewarded with credits).  
	    
	    \subsubsection{Teachers / Assistant Teachers}
	    The teachers and assistant teachers can both represent someone needing help or someone answering questions. These
	    stakeholders activities on the platform will mainly include writing questions and answers. He's probably between 
	    18-70 years old and can be from any country in the world. This stakeholder should have sufficient knowledge in the 
	    domaine of web application usage as part of teaching requires you to manager resources distribution through web 
	    applications such as Moodle~\cite{moodleWebsite}. The success criteria for these stakeholders is to be able to 
	    share knowledge about their domains widely and easily. The expected deliverable by the teaching corpus is a web 
	    app working as a mean of answering questions and provide solutions, tips and tricks regarding their courses 
	    contents.
	    
	    \subsubsection{Institution Administrators}
        The institution administrators represent the link between our platform and institutions. It's the person in charge
        of completing and maintaining up to date the informations and settings of the institution they represent. They are
        responsible of the correctness of each of these informations. They are also in responsible to ensure that students
        with sufficient in-app credits can exchange them with academic credits easily. The success criteria for institution
        administrators is for them to be able to simply manage and offer great sharing features to the institutions students.
        The deliverable expected by these stakeholders is also the final web app. These stakeholders are the most important
        of the project as they remain the only link between the platform and the institutions. 
	    
	\subsection{Users \& Stakeholders Key Requirements}
	
	\textbf{Problems}
	\begin{itemize}
	    \item Difficulty to get local help in specific domain / course
	\end{itemize}

	    	
	\subsection{Competition and Alternatives}
	
% PRODUCT OVERVIEW SECTION $$$$$$$$$$$$$$$$$$$$$$$$$$$$$$$$$$$$$$$$$$$$$$$$$$$$$$$$$$$$$$$$$$$$$$$$
    \newpage
	\section{Product Overview}
	\subsection{Product Perspective}
	\subsection{Features Summary}
	\subsection{Hypothesis}
	\subsection{User Environment}
	\subsection{Cost \& Pricing}
	\subsection{Licences \& Installation}
	
% PRODUCT ESSENTIALS FEATURES SECTION $$$$$$$$$$$$$$$$$$$$$$$$$$$$$$$$$$$$$$$$$$$$$$$$$$$$$$$$$$$$$$$$$$$$$$$$
    \newpage
	\section{Product Essentials Features}

% PRODUCT CONSTRAINTS SECTION $$$$$$$$$$$$$$$$$$$$$$$$$$$$$$$$$$$$$$$$$$$$$$$$$$$$$$$$$$$$$$$$$$$$$$$$
    \newpage
	\section{Product Constraints}
	
% NON-FUNCTIONAL QUALITY TOLERANCES SECTION $$$$$$$$$$$$$$$$$$$$$$$$$$$$$$$$$$$$$$$$$$$$$$$$$$$$$$$$$$$$$$$$$$$$$$$$
    \newpage
	\section{Non-Functional Quality Tolerances}
	
% MUTUAL FEATURES PRIORITY SECTION $$$$$$$$$$$$$$$$$$$$$$$$$$$$$$$$$$$$$$$$$$$$$$$$$$$$$$$$$$$$$$$$$$$$$$$$
	\newpage
	\section{Mutual Features Priority}
	
% OTHER PRODUCT REQUIREMENTS SECTION $$$$$$$$$$$$$$$$$$$$$$$$$$$$$$$$$$$$$$$$$$$$$$$$$$$$$$$$$$$$$$$$$$$$$$$$
    \newpage
	\section{Other Product Requirements}
	\subsection{Applicable Standards}
	\subsection{System Requirements}
	\subsection{Performance}
	\subsection{Environment-Related Requirements}

% DOCUMENTATION REQUIREMENTS SECTION $$$$$$$$$$$$$$$$$$$$$$$$$$$$$$$$$$$$$$$$$$$$$$$$$$$$$$$$$$$$$$$$$$$$$$$$
    \newpage
	\section{Documentation Requirements}
	\subsection{User Manual}
	\subsection{Online Assistance}
	\subsection{Installation \& Configuration Guide}
	\subsection{Packaging, Labelling \& Copyright}
    \newpage
    \bibliography{refs}{}
    \bibliographystyle{plain}
\end{document}
